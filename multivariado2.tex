% Options for packages loaded elsewhere
\PassOptionsToPackage{unicode}{hyperref}
\PassOptionsToPackage{hyphens}{url}
%
\documentclass[
]{book}
\usepackage{amsmath,amssymb}
\usepackage{lmodern}
\usepackage{iftex}
\ifPDFTeX
  \usepackage[T1]{fontenc}
  \usepackage[utf8]{inputenc}
  \usepackage{textcomp} % provide euro and other symbols
\else % if luatex or xetex
  \usepackage{unicode-math}
  \defaultfontfeatures{Scale=MatchLowercase}
  \defaultfontfeatures[\rmfamily]{Ligatures=TeX,Scale=1}
\fi
% Use upquote if available, for straight quotes in verbatim environments
\IfFileExists{upquote.sty}{\usepackage{upquote}}{}
\IfFileExists{microtype.sty}{% use microtype if available
  \usepackage[]{microtype}
  \UseMicrotypeSet[protrusion]{basicmath} % disable protrusion for tt fonts
}{}
\makeatletter
\@ifundefined{KOMAClassName}{% if non-KOMA class
  \IfFileExists{parskip.sty}{%
    \usepackage{parskip}
  }{% else
    \setlength{\parindent}{0pt}
    \setlength{\parskip}{6pt plus 2pt minus 1pt}}
}{% if KOMA class
  \KOMAoptions{parskip=half}}
\makeatother
\usepackage{xcolor}
\usepackage{graphicx}
\makeatletter
\def\maxwidth{\ifdim\Gin@nat@width>\linewidth\linewidth\else\Gin@nat@width\fi}
\def\maxheight{\ifdim\Gin@nat@height>\textheight\textheight\else\Gin@nat@height\fi}
\makeatother
% Scale images if necessary, so that they will not overflow the page
% margins by default, and it is still possible to overwrite the defaults
% using explicit options in \includegraphics[width, height, ...]{}
\setkeys{Gin}{width=\maxwidth,height=\maxheight,keepaspectratio}
% Set default figure placement to htbp
\makeatletter
\def\fps@figure{htbp}
\makeatother
\setlength{\emergencystretch}{3em} % prevent overfull lines
\providecommand{\tightlist}{%
  \setlength{\itemsep}{0pt}\setlength{\parskip}{0pt}}
\setcounter{secnumdepth}{-\maxdimen} % remove section numbering
\ifLuaTeX
  \usepackage{selnolig}  % disable illegal ligatures
\fi
\IfFileExists{bookmark.sty}{\usepackage{bookmark}}{\usepackage{hyperref}}
\IfFileExists{xurl.sty}{\usepackage{xurl}}{} % add URL line breaks if available
\urlstyle{same} % disable monospaced font for URLs
\hypersetup{
  pdftitle={Análisis Multivariado 2},
  pdfauthor={Estudiante: Mauro Loprete, Profesora: Natalia Da Silva},
  hidelinks,
  pdfcreator={LaTeX via pandoc}}

\title{Análisis Multivariado 2}
\usepackage{etoolbox}
\makeatletter
\providecommand{\subtitle}[1]{% add subtitle to \maketitle
  \apptocmd{\@title}{\par {\large #1 \par}}{}{}
}
\makeatother
\subtitle{Carpeta de ejercicios}
\author{Estudiante: Mauro Loprete, Profesora: Natalia Da Silva}
\date{2022-09-27}

\begin{document}
\frontmatter
\maketitle

\mainmatter
\hypertarget{sobre-esta-puxe1gina}{%
\chapter{Sobre esta página}\label{sobre-esta-puxe1gina}}

La idea principal es jugar con bookdown y mostrar los resultados
obtenidos en la lista de ejercicios de la materia. La mayoría de
ejercicios, provienen del libro
\href{https://www.statlearning.com/}{Introudction to Statistical
Learning 2nd Ed.}:

\begin{figure}
\centering
\includegraphics{https://images-na.ssl-images-amazon.com/images/I/61Lvnv9+CML.jpg}
\caption{islr}
\end{figure}

\hypertarget{statistical-learning}{%
\chapter{Statistical learning}\label{statistical-learning}}

Placeholder

\hypertarget{ejercicio-extra}{%
\section*{Ejercicio extra}\label{ejercicio-extra}}
\addcontentsline{toc}{section}{Ejercicio extra}

\hypertarget{ejercicio-1}{%
\section*{Ejercicio 1}\label{ejercicio-1}}
\addcontentsline{toc}{section}{Ejercicio 1}

\hypertarget{ejercicio-5}{%
\section*{Ejercicio 5}\label{ejercicio-5}}
\addcontentsline{toc}{section}{Ejercicio 5}

\hypertarget{ejercicio-6}{%
\section*{Ejercicio 6}\label{ejercicio-6}}
\addcontentsline{toc}{section}{Ejercicio 6}

\hypertarget{linear-regression}{%
\chapter{Linear regression}\label{linear-regression}}

\hypertarget{ejercicio-4}{%
\section*{Ejercicio 4}\label{ejercicio-4}}
\addcontentsline{toc}{section}{Ejercicio 4}

\emph{I collect a set of data (n = 100 observations) containing a single
predictor and a quantitative response. I then fit a linear regression
model to the data, as well as a separate cubic regression,
i.e.~\(Y = \beta_{0} + \beta_{1}X + \beta_{2}X^{2} + \beta_{3}X^{3} + \varepsilon\).}

\begin{itemize}
\item
  \textbf{(a) Suppose that the true relationship between X and Y is
  linear, i.e.~\(Y = \beta_{0} + \beta_{1}X + \varepsilon\). Consider
  the training residual sum of squares (\(RSS\)) for the linear
  regression, and also the training \(RSS\) for the cubic regression.
  Would we expect one to be lower than the other, would we expect them
  to be the same, or is there not enough information to tell? Justify
  your answer.}

  Al que estamos bajo el supuesto que la relación entre la variable
  respuesta es exactamente lineal y que el la esperanza del RSS lo
  podemos descomponer en :

  \begin{equation*}
        E(RSS) = E\left[\left(Y - \hat{Y}\right)^{2}\right] = E\left[f(X) + \varepsilon - \hat{f}(X)\right]^{2} = \underbrace{\left[f(X) - \hat{f}(X)\right]^{2}}_{\longrightarrow 0} + Var(\varepsilon)
    \end{equation*}

  El primer sumando tiende a cero, ya que la relación es exactamente
  lineal y en el caso que nuestra \(\hat{f}(X)\) considere una forma
  cúbica el sesgo por la especificación sería mayor que en este caso.
\item
  \textbf{(b) Answer (a) using test rather than training \(RSS\).}

  Como se comento anteriormente, se esperaría que el \(RSS\) de training
  del modelo de regresión lineal sea mas bajo que el del modelo cúbico,
  debido a que la verdadera relación es lineal. El modelo cúbico al
  intentar ajustar con una especificación incorrecta es mas sensible a
  cambios en la muestra, por lo tanto esperaría una mayor varianza en el
  término de error.
\item
  \textbf{(c) Suppose that the true relationship between X and Y is not
  linear, but we don't know how far it is from linear. Consider the
  training \(RSS\) for the linear regression, and also the training
  \(RSS\) for the cubic regression. Would we expect one to be lower than
  the other, would we expect them to be the same, or is there not enough
  information to tell? Justify your answer.}

  En este caso, el modelo cúbico tendría un menor \(RSS\) de training
  debido a que es un modelo mas flexible ya que seguira mas de cerca los
  puntos.
\item
  \textbf{(d) Answer (c) using test rather than training \(RSS\).}

  Si bien sabemos que la relación no es lineal, no sabemos que órden
  tiene la función polinomial que generan estos datos, modelos no
  lineales tendrían un menor sesgo por su flexibilidad aunque este puede
  crecer de forma considerable en el término de la varianza del ruido si
  el órden no es el correcto por más que sea un error sumamente
  saturado.
\end{itemize}

\hypertarget{moving-beyond-linearity}{%
\chapter{Moving Beyond Linearity}\label{moving-beyond-linearity}}

Placeholder

\hypertarget{ejercicio-1-1}{%
\section*{Ejercicio 1}\label{ejercicio-1-1}}
\addcontentsline{toc}{section}{Ejercicio 1}

\hypertarget{ejercicio-2}{%
\section*{Ejercicio 2}\label{ejercicio-2}}
\addcontentsline{toc}{section}{Ejercicio 2}

\hypertarget{ejercicio-3}{%
\section*{Ejercicio 3}\label{ejercicio-3}}
\addcontentsline{toc}{section}{Ejercicio 3}

\hypertarget{ejercicio-4-1}{%
\section*{Ejercicio 4}\label{ejercicio-4-1}}
\addcontentsline{toc}{section}{Ejercicio 4}

\hypertarget{ejercicio-5-1}{%
\section*{Ejercicio 5}\label{ejercicio-5-1}}
\addcontentsline{toc}{section}{Ejercicio 5}

\hypertarget{references}{%
\chapter*{References}\label{references}}
\addcontentsline{toc}{chapter}{References}

\hypertarget{moving-beyond-linearity-1}{%
\chapter{Moving Beyond Linearity}\label{moving-beyond-linearity-1}}

\begin{verbatim}
library(ISLR)

mod <- 
  linear_reg() %>% 
  set_engine("lm")

recipe <- recipe(
    wage ~ age,
    data = Wage
) %>%
step_bs(
    age,
    options = list(
        knots = c(25,40,60),
        intercept = FALSE
    )
)

workflow() %>%
    add_model(mod) %>%
    add_recipe(recipe) %>%
    fit(
        data = Wage
    ) %>%
    assign(
        "model",
        .,
        envir = .GlobalEnv
    )


age.grid <- seq(
    from = min(Wage$age), 
    to = max(Wage$age)
)


predict(
    model,
    new_data = list(
        age = age.grid
    ),
    se = TRUE
)

library (splines)
fit <- lm(wage ~ bs(age , knots = c(25, 40, 60)), data = Wage)
pred <- predict (fit , newdata = list (age = age.grid), se = T)
plot (Wage$age , Wage$wage , col = " gray ")
lines (age.grid, pred$fit , lwd = 2)
lines (age.grid , pred$fit + 2 * pred$se, lty = "dashed")
lines (age.grid , pred$fit - 2 * pred$se, lty = "dashed"")
\end{verbatim}

\backmatter
\end{document}
